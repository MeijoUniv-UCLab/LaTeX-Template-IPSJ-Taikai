%--------------------------------------------------------------
% ipsj-zenkoku-template.tex
% 情報処理学会全国大会原稿 非公式LaTeXテンプレートファイル
%
% @author: Hidekazu Suzuki (hsuzuki@meijo-u.ac.jp)
% @version: 1.2 (19 Dec 2023)
%--------------------------------------------------------------

%% 本文のフォントサイズ=10pt,2段組
\documentclass[a4j,10pt,twocolumn,uplatex]{jsarticle}

\usepackage{ipsj-zenkoku}	% 全国大会用パッケージ

%% 和文題目
\title{IPSJ全国大会論文フォーマット(タイトル)}
%% 和文著者
\author{氏 名\DAG{1} \quad 氏 名\DAG{2} \quad 氏 名\DAG{3}}
%% 和文所属
\affiliation{所属\DAG{1} \qquad 所属\DAG{2} \qquad 所属\DAG{3}}

%% 英文題目
\etitle{IPSJ National Convention Paper Format (Title)}
%% 英文著者・所属
\eauthor{%
	\DAG{1} Firstname Lastname, Department, University\\
	\DAG{2} Firstname Lastname, Department, University\\
    \DAG{3} Firstname Lastname, Department, University
}

\begin{document}

\maketitle		% 和文タイトルの出力
\makeetitle		% 英文タイトルの出力


% 本文の行間に設定
\setlength{\baselineskip}{1.5zh}


\section{はじめに}

毎年3月に開催される情報処理学会全国大会は,原稿のテンプレートがMicrosoft Wordしか公開されていない.
そこで,Wordテンプレートをもとに,非公式\LaTeX テンプレートを作成した.
本稿では本テンプレートの使い方を解説する.


\section{ソースファイルの構成}

\subsection{パッケージ}
本テンプレートのスタイルを定義している\verb|ipsj-zenkoku|を必ず利用すること.
その他に利用するパッケージがある場合は,適宜\verb|\usepackage{xxx}|で追加すること.

\subsection{プリアンブル}
\begin{itemize}
    \item \verb|\title|,\verb|\etitle|:和文表題,英文表題
    \item \verb|\author|,\verb|\affiliation|:和文著者名,和文所属
    \item \verb|\eauthor|:英文著者名および英文所属.
    1行に著者1名の情報をを記入し,\verb|\\|で改行すること.
\end{itemize}

\subsection{タイトルの表示}
\verb|\maketitle|および\verb|\makeetitle|によりタイトル(題目,著者,所属)および脚注の英文表記が出力されるため,消さないように.

\subsection{本文}
IPSJ全国大会の原稿はA4で2ページ,かつファイルサイズは2MB以下である必要がある.

\subsection{箇条書き}
番号無し箇条書きは,下記のように出力される.
\begin{itemize}
    \item 項目1
    \item 項目2
\end{itemize}

番号付き箇条書きは,括弧付きで表示されるようにスタイルファイルで設定している.
\begin{enumerate}
    \item 項目1
    \item 項目2
\end{enumerate}

\subsection{図}
\figref{fig:tiger}のように,PNG形式のほか,PDF形式の図形ファイルを取り込むことができる.

\begin{figure}[tb]
    \centering
    \includegraphics[scale=1.0,clip]{fig/tiger.png}
    \caption{トラ}
    \label{fig:tiger}
\end{figure}

\subsection{表}
本テンプレートでは,情報処理学会論文誌の書き方に準拠して,\tabref{tab:sample}のように罫線を少なくして仕上がりをスッキリさせている.
図と同じく,キャプションは英文で記載する.
下記の点に気をつけて表を作成すること.
\begin{itemize}
    \item 表の最上部の罫線は\verb|\hline\hline|として二重線とする.
    \item 表の最下部は一重線とする.
    \item その他の罫線は見出しとデータの境界などに限定する.
\end{itemize}

\begin{table}[tb]
    \centering
    \footnotesize	% 表中のフォントサイズをfootnotesizeに設定(必ず指定すること)
    \caption{サンプル}
    \label{tab:sample}
    \begin{tabular}{l|lll}
        \hline\hline
         & ヘッダ1 & ヘッダ2 & ヘッダ3 \\ 
        \hline 
        項目1 & データ11 & データ12 & データ13 \\
        項目2 & データ22 & データ22 & データ22 \\  
        \hline 
    \end{tabular}
\end{table}

\subsection{図表の参照と配置}
本文から図表を参照する場合は,情報処理学会論文誌の\LaTeX テンプレートで使われる下記のマクロを利用する.
\begin{itemize}
    \item \verb|\figref{x}|:\verb|\label{x}|を設定した図の参照
    \item \verb|\tabref{y}|:\verb|\label{y}|を設定した表の参照
\end{itemize}

図および表は段落の途中で掲載するのではなく,ページ上部か下部のどちらかに寄せて配置する.
すなわち,\verb|\begin{figure}[z]|および\verb|\begin{table}[z]|の\texttt{z}の部分には,``\texttt{t}''(上部)または``\texttt{b}''(下部)のいずれかとする.

\subsection{参考文献}
参考文献は最後の\verb|thebibliography|環境に記載する.
文献情報は\verb|\bibitem{label}|の後に,著者名,掲載誌名,巻,号,ページ,発行年などを入力する.
書き方の一例として,論文誌\cite{Matsuoka22},国際会議\cite{Suzuki13},RFC\cite{MIPv4}を示す.


\section{まとめ}

本稿は非公式\LaTeX テンプレートに基づいて作成されている.
本稿のソースファイルをコピーして必要な箇所を修正すれば,公式テンプレートのフォーマットにほぼ準拠した原稿PDFを作成できるため,是非利用してほしい.


\subsection*{謝辞}
謝辞を記載する必要がある場合は,ここに記載する.
不要であれば\verb|\subsection*{謝辞}|ごと削除する.


%% 参考文献
\begin{thebibliography}{9}	% ←文献数が1桁なら9,2桁なら99を指定
    \bibitem{Matsuoka22} 松岡 穂,鈴木秀和,内藤克浩:拡張NTMobileを用いたアプリケーションレベルで実現するシームレスIP Flow Mobility,情報処理学会論文誌, Vol.~63, No.~1, pp.~130--142, 2022.
    \bibitem{Suzuki13} H. Suzuki, K. Naito, K. Kamienoo, et al.: NTMobile: new end-to-end communication architecture in IPv4 and IPv6 networks, Proc. of ACM MobiCom 2013, pp.~171--174, 2013.
    \bibitem{MIPv4} C. Perkins: RFC 5944, IETF, 2010.
\end{thebibliography}

\end{document}